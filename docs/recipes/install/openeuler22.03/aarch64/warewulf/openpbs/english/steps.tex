\documentclass[letterpaper]{article}
\usepackage{common/ohpc-doc}
\setcounter{secnumdepth}{5}
\setcounter{tocdepth}{5}

% Include git variables
\input{vc.tex}

% Define Base OS and other local macros
\newcommand{\baseOS}{openEuler 22.03}
\newcommand{\OSRepo}{openEuler\_22.03}
\newcommand{\OSTree}{openEuler\_22.03}
\newcommand{\OSTag}{oe2203}
\newcommand{\baseos}{openeuler22.03}
\newcommand{\baseosshort}{openeuler22.03}
\newcommand{\provisioner}{Warewulf}
\newcommand{\provheader}{\provisioner{}}
\newcommand{\rms}{OpenPBS}
\newcommand{\rmsshort}{OpenPBS}
\newcommand{\arch}{aarch64}

% Define package manager commands
\newcommand{\pkgmgr}{yum}
\newcommand{\addrepo}{wget -P /etc/yum.repos.d}
\newcommand{\chrootaddrepo}{wget -P \$CHROOT/etc/yum.repos.d}
\newcommand{\clean}{yum clean expire-cache}
\newcommand{\chrootclean}{yum --installroot=\$CHROOT clean expire-cache}
\newcommand{\install}{yum -y install}
\newcommand{\chrootinstall}{yum -y --installroot=\$CHROOT install}
\newcommand{\groupinstall}{yum -y groupinstall}
\newcommand{\groupchrootinstall}{yum -y --installroot=\$CHROOT groupinstall}
\newcommand{\remove}{yum -y remove}
\newcommand{\upgrade}{yum -y upgrade}
\newcommand{\chrootupgrade}{yum -y --installroot=\$CHROOT upgrade}
\newcommand{\tftppkg}{syslinux-tftpboot}

% boolean for os-specific formatting
\toggletrue{isaarch}
\toggletrue{isCentOS}
\toggletrue{isCentOS_ww_pbs_aarch}
\toggletrue{ispbs}
\toggletrue{isWarewulf}

\begin{document}
\graphicspath{{common/figures/}}
\thispagestyle{empty}

% Title Page --------------------------------------------------------
\input{common/title}
% Disclaimer
\newpage

\vspace*{3.0cm}
\noindent {\Large \color{logoblue} \fontfamily{phv}\selectfont Legal Notice} \\ 

\vspace*{0.5cm}

\noindent Copyright {\small\copyright} 2016-2023, OpenHPC, a Linux Foundation
Collaborative Project. All rights reserved. \\

\vspace*{0.1cm}

\noindent \begin{tabular}{cp{10cm}}
    {\includegraphics[width=0.22\textwidth]{../common/figures/cc_by.pdf}}
    %\raisebox{-.75\height}{\includegraphics[width=0.22\textwidth]{../../common/figures/cc_by.pdf}} &
This documentation is licensed under the Creative Commons Attribution 4.0 International
License. To view a copy of this license, visit
\href{http://creativecommons.org/licenses/by/4.0}{\color{blue}{http://creativecommons.org/licenses/by/4.0}}. \\
\end{tabular}


\vspace*{1.5cm}

{\footnotesize

\noindent Intel, the Intel logo, and other Intel marks are trademarks of Intel
Corporation in the U.S. and/or other countries. \\
\iftoggleverb{ispbs}
\noindent Altair, the Altair logo, OpenPBS, and other Altair marks are
trademarks of Altair Engineering, Inc. in the U.S. and/or other countries. \\
\fi
\noindent *Other names and brands may be claimed as the property of others. \\

}
 

\newpage
\tableofcontents
\newpage

% Introduction  ----------------------------------------------------


\section{Introduction} \label{sec:introduction}
\input{common/install_header}
\input{common/intro} \\

\input{common/base_edition/edition}
\input{common/audience}
\input{common/requirements}
\input{common/inputs}

% begin_ohpc_run
% ohpc_validation_newline
% ohpc_validation_comment Verify OpenHPC repository has been enabled before proceeding
% ohpc_validation_newline
% ohpc_command yum repolist | grep -q OpenHPC
% ohpc_command if [ $? -ne 0 ];then
% ohpc_command    echo "Error: OpenHPC repository must be enabled locally"
% ohpc_command    exit 1
% ohpc_command fi
% end_ohpc_run

% Base Operating System --------------------------------------------

\section{Install Base Operating System (BOS)}
\input{common/bos}

% begin_ohpc_run
% ohpc_validation_newline
% ohpc_validation_comment Disable firewall 
\begin{lstlisting}[language=bash,keywords={}]
[sms](*\#*) systemctl disable firewalld
[sms](*\#*) systemctl stop firewalld
\end{lstlisting}
% end_ohpc_run

% ------------------------------------------------------------------

\section{Install \OHPC{} Components} \label{sec:basic_install}
\input{common/install_ohpc_components_intro.tex}

\subsection{Enable \OHPC{} repository for local use} \label{sec:enable_repo}
\input{common/enable_ohpc_repo}
In addition to the \OHPC{} 
\iftoggle{isxCAT}{and \xCAT{} package repositories,}{package repository,}
the {\em master} host also requires access to the standard base OS distro
repositories in order to resolve necessary dependencies. For \baseOS{}, the
requirements are to have access to the {\color{purple}{OS}}, {\color{purple}{Everything}},
{\color{purple}{EPOL main}} and {\color{purple}{EPOL update}} repositories
for which mirrors are freely available online:

\begin{itemize*}
\item OS
  (e.g. \href{http://repo.openeuler.org/openeuler/openEuler-22.03-LTS-SP2/OS/}
             {\color{blue}{http://repo.openeuler.org/openeuler/openEuler-22.03-LTS-SP2/OS/}} )
\item Everything
  (e.g. \href{http://repo.openeuler.org/openeuler/openEuler-22.03-LTS-SP2/everything/}
            {\color{blue}{repo.openeuler.org/openeuler/openEuler-22.03-LTS-SP2/everything/}} )
\item EPOL-main
  (e.g. \href{http://repo.openeuler.org/openeuler/openEuler-22.03-LTS-SP2/EPOL/main/}
            {\color{blue}{http://repo.openeuler.org/openeuler/openEuler-22.03-LTS-SP2/EPOL/main/}} )
\item EPOL-update
  (e.g. \href{http://repo.openeuler.org/openeuler/openEuler-22.03-LTS-SP2/EPOL/update/main}
            {\color{blue}{http://repo.openeuler.org/openeuler/openEuler-22.03-LTS-SP2/EPOL/update/main}} )
\end{itemize*}

One might replace {\color{blue}{http://repo.openeuler.org}} with any of the mirrors listed
\href{https://www.openeuler.org/en/mirror/list/}{\color{blue}{here}}

\input{common/automation}

\subsection{Add provisioning services on {\em master} node} \label{sec:add_provisioning}
\input{common/install_provisioning_intro}
\input{common/enable_pxe}
\input{common/time}

\subsection{Add resource management services on {\em master} node} \label{sec:add_rm}
\input{common/install_openpbs}

%% Add if/when IB is available for testing
%% \subsection{Optionally add \InfiniBand{} support services on {\em master} node} \label{sec:add_ofed}
%% \input{common/ibsupport_sms_centos}

%\vspace*{-0.15cm}
\subsection{Complete basic Warewulf setup for {\em master} node} \label{sec:setup_ww}
\input{common/warewulf_setup}
\input{common/warewulf_setup_centos}

\subsection{Define {\em compute} image for provisioning}
With the provisioning services enabled, the next step is to define and
customize a system image that can subsequently be used to provision one or more
{\em compute} nodes. The following subsections highlight this process.

\subsubsection{Build initial BOS image} \label{sec:assemble_bos}
The \OHPC{} build of \Warewulf{} includes specific enhancements enabling support for
\baseOS{}. The following steps illustrate the process to build a minimal, default
image for use with \Warewulf{}. We begin by defining a directory structure on the 
{\em master} host that will represent the root filesystem of the compute node. The 
default location for this example is in
\texttt{/opt/ohpc/admin/images/\baseos{}}.

\begin{center}
  \begin{tcolorbox}[]
    \small \Warewulf{} is configured by default to access an external
    repository (repo.openeuler.org) during the \texttt{wwmkchroot} process.  If
    the master host cannot reach the public openEuler mirrors, or if you prefer to
    access a locally cached mirror, set the \texttt{\$\{YUM\_MIRROR\}}
    environment variable to your desired repo location {\em prior} to running
    the \texttt{wwmkchroot} command below. For example:

% begin_ohpc_run
% ohpc_command if [ ! -z ${BOS_MIRROR+x} ]; then
% ohpc_indent 5
\begin{lstlisting}[language=bash,keywords={}]
# Override default OS repository (optional) - set YUM_MIRROR variable to desired repo location
[sms](*\#*) export YUM_MIRROR=${BOS_MIRROR}
\end{lstlisting}
% ohpc_indent 0
% ohpc_command fi
% end_ohpc_runƒ

\end{tcolorbox}
\end{center}

% begin_ohpc_run
% ohpc_comment_header Create compute image for Warewulf \ref{sec:assemble_bos}
\begin{lstlisting}[language=bash,literate={-}{-}1,keywords={},upquote=true,keepspaces,literate={BOSVER}{\baseos{}}1]
# Define chroot location 
[sms](*\#*) export CHROOT=/opt/ohpc/admin/images/BOSVER

# Build initial chroot image
[sms](*\#*) wwmkchroot -v openeuler-22.03 $CHROOT
# Enable OpenHPC and openEuler repos inside chroot
[sms](*\#*) dnf -y --installroot $CHROOT install openEuler-release
[sms](*\#*) cp -p /etc/yum.repos.d/OpenHPC*.repo $CHROOT/etc/yum.repos.d
\end{lstlisting}
% end_ohpc_run


\subsubsection{Add \OHPC{} components} \label{sec:add_components}
\input{common/add_to_compute_chroot_intro}

% begin_ohpc_run
% ohpc_validation_comment Add OpenHPC components to compute instance
\begin{lstlisting}[language=bash,literate={-}{-}1,keywords={},upquote=true]
# Add OpenPBS client support
[sms](*\#*) (*\chrootinstall*) openpbs-execution-ohpc
[sms](*\#*) perl -pi -e "s/PBS_SERVER=\S+/PBS_SERVER=${sms_name}/" $CHROOT/etc/pbs.conf
[sms](*\#*) echo "PBS_LEAF_NAME=${sms_name}" >> /etc/pbs.conf
[sms](*\#*) chroot $CHROOT opt/pbs/libexec/pbs_habitat
[sms](*\#*) perl -pi -e "s/\$clienthost \S+/\$clienthost ${sms_name}/" $CHROOT/var/spool/pbs/mom_priv/config
[sms](*\#*) echo "\$usecp *:/home /home" >> $CHROOT/var/spool/pbs/mom_priv/config
[sms](*\#*) chroot $CHROOT systemctl enable pbs

# Add Network Time Protocol (NTP) support
[sms](*\#*) (*\chrootinstall*) chrony
[sms](*\#*) echo "server ${sms_ip} iburst" >> $CHROOT/etc/chrony.conf

# Add kernel drivers (matching kernel version on SMS node)
[sms](*\#*) (*\chrootinstall*) kernel-`uname -r`

# Include modules user environment
[sms](*\#*) (*\chrootinstall*) lmod-ohpc
\end{lstlisting}
% end_ohpc_run

\subsubsection{Customize system configuration} \label{sec:master_customization}
\input{common/warewulf_chroot_customize_centos}
\input{common/restart_nfs}

%\clearpage
\subsubsection{Additional Customization ({\em optional})} \label{sec:addl_customizations}
\input{common/compute_customizations_intro}

%% Add if/when IB is available for testing
%% \paragraph{Increase locked memory limits}
%% \input{common/memlimits}

%%\paragraph{Add \Lustre{} client} \label{sec:lustre_client}
%%\input{common/lustre-client}
%%\input{common/lustre-client-centos}
%%\vspace*{0.5cm}
%%\input{common/lustre-client-post}

%\clearpage
\vspace*{-.2cm}
\paragraph{Enable forwarding of system logs} \label{sec:add_syslog}
\input{common/syslog}

%\vspace*{-.2cm}
\clearpage
\paragraph{Add \Nagios{} monitoring} \label{sec:add_nagios}
\input{common/nagios}

\clearpage
%\vspace*{-0.32cm}
\paragraph{Add \clustershell{}}
\input{common/clustershell}

%\clearpage
%\paragraph{Add \mrsh{}}
%\input{common/mrsh}

\paragraph{Add \genders{}}
\input{common/genders}

\vspace*{-.1cm}
\paragraph{Add Magpie}
\input{common/magpie}

\vspace*{-.1cm}
\paragraph{Add \conman{}} \label{sec:add_conman}
\input{common/conman}

\vspace*{-.1cm}
\paragraph{Add \nhc{}} \label{sec:add_nhc}
\input{common/nhc}

\subsubsection{Import files} \label{sec:file_import}
\input{common/import_ww_files}
%% \input{common/import_ww_files_ib_centos}
\input{common/finalize_provisioning}
\vspace*{0.2cm}
\input{common/add_ww_hosts_intro}
\input{common/add_ww_hosts_pbs}
\input{common/add_ww_hosts_finalize}

\vspace*{-0.2cm}
\subsubsection{Optional kernel arguments} \label{sec:optional_kargs}
\input{common/conman_post}
\input{common/kargs_post}

\vspace*{-0.2cm}
\subsubsection{Optionally configure stateful provisioning}
\input{common/stateful}

\vspace*{-0.1cm}
\subsection{Boot compute nodes} \label{sec:boot_computes}
\input{common/reset_computes} 

\vspace*{-0.50cm}
\section{Install \OHPC{} Development Components} \label{sec:install_dev}
\input{common/dev_intro.tex}

\vspace*{-0.15cm}
\subsection{Development Tools} \label{sec:install_dev_tools}
\input{common/dev_tools}

\vspace*{-0.15cm}
\subsection{Compilers} \label{sec:install_compilers}
\input{common/compilers}

%\clearpage
\vspace*{0.5cm}
\subsection{MPI Stacks} \label{sec:mpi}
\input{common/mpi_aarch_openpbs}

\subsection{Performance Tools} \label{sec:install_perf_tools}
\input{common/perf_tools}

\subsection{Setup default development environment}
\input{common/default_dev}

%\clearpage
\subsection{3rd Party Libraries and Tools} \label{sec:3rdparty}
\input{common/third_party_libs_intro}
\input{common/third_party_libs_petsc_centos}
\vspace*{.4cm}
\input{common/third_party_libs}
\input{common/third_party_mpi_libs_aarch}

\subsection{Optional Development Tool Builds} \label{sec:3rdparty_arm}
\input{common/armhpc_enabled_builds}

\clearpage
\section{Resource Manager Startup} \label{sec:rms_startup}
\input{common/openpbs_startup}

\section{Run a Test Job} \label{sec:test_job}
\input{common/openpbs_test_job}

\clearpage
\appendix
{\bf \LARGE \centerline{Appendices}} \vspace*{0.2cm}

\addcontentsline{toc}{section}{Appendices}
\renewcommand{\thesubsection}{\Alph{subsection}}

\input{common/automation_appendix}
\input{common/upgrade}
\input{common/test_suite}
\input{common/customization_appendix_centos}
\input{common/manifest_intro}

\newcommand{\firstColWidth}{5.5cm}
\newcommand{\secondColWidth}{0.95cm}

\vspace*{1.0cm}
\urlstyle{same}

% Administration Tools 
\begin{table}[h]
\caption{\bf Administrative Tools} \vspace*{\captionSpace{}} \label{table:admin}
\input manifest/admin
\end{table}
\vspace*{0.5cm}

\renewcommand{\firstColWidth}{6.25cm}
\renewcommand{\secondColWidth}{0.95cm}

% Provisioning
\begin{table}[h!]
\caption{\bf Provisioning} \vspace*{\captionSpace{}} \label{table:provisioning}
\input manifest/provisioning
\vspace*{\tabSpaceBot{}}
\end{table}

\renewcommand{\firstColWidth}{4.1cm}
\renewcommand{\secondColWidth}{1.8cm}

% Resource Management
\begin{table}[h!]
\caption{\bf Resource Management} \vspace*{\captionSpace{}} \label{table:rms}
\input manifest/rms
\vspace*{\tabSpaceBot{}}
\end{table}

\renewcommand{\firstColWidth}{4.1cm}

% Compiler Families
\begin{table}[h!]
\caption{\bf Compiler Families} \vspace*{\captionSpace{}} \label{table:compiler-families}
\input manifest/compiler-families
\vspace*{\tabSpaceBot{}}
\end{table}

\renewcommand{\firstColWidth}{5.4cm}
\renewcommand{\secondColWidth}{1.5cm}

% MPI Families
\begin{table}[h!]
\caption{\bf MPI Families / Communication Libraries} \vspace*{\captionSpace{}} \label{table:mpi-families}
\input manifest/mpi-families
\vspace*{\tabSpaceBot{}}
\end{table}

\renewcommand{\firstColWidth}{5.4cm}
\renewcommand{\secondColWidth}{1.5cm}

% Development Tools
\begin{table}[h!]
\caption{\bf Development Tools} \vspace*{\captionSpace{}} \label{table:dev-tools}
\input manifest/dev-tools
\vspace*{\tabSpaceBot{}}
\end{table}

\renewcommand{\firstColWidth}{4.4cm}
\renewcommand{\secondColWidth}{1.72cm}

% Perf Tools
\begin{table}[h!]
\caption{\bf Performance Analysis Tools} \vspace*{\captionSpace{}} \label{table:perf-tools}
\input manifest/perf-tools
\vspace*{\tabSpaceBot{}}
\end{table}

%%% % Distro Packages
%%% \begin{table}[h!]
%%% \caption{\bf Distro Support Packages/Dependencies} \vspace*{\captionSpace{}} \label{table:distro-packages}
%%% \input manifest/distro-packages
%%% \vspace*{\tabSpaceBot{}}
%%% \end{table}

\renewcommand{\firstColWidth}{5.05cm}
\renewcommand{\secondColWidth}{1.2cm}

%%% % Lustre
%%% \begin{table}[h!]
%%% \caption{\bf Lustre} \vspace*{\captionSpace{}} \label{table:lustre}
%%% \input manifest/lustre
%%% \vspace*{\tabSpaceBot{}}
%%% \end{table}

% IO Libs
\begin{table}[h!]
\caption{\bf IO Libraries} \vspace*{\captionSpace{}} \label{table:io-libs}
\input manifest/io-libs
\vspace*{\tabSpaceBot{}}
\end{table}

%\renewcommand{\firstColWidth}{4.5cm}
%\renewcommand{\secondColWidth}{1.5cm}

% Runtimes
\clearpage
\begin{table}[h!]
\caption{\bf Runtimes} \vspace*{\captionSpace{}} \label{table:runtimes}
\input manifest/runtimes
\vspace*{\tabSpaceBot{}}
\end{table}

% Serial libs
\begin{table}[h!]
\caption{\bf Serial/Threaded Libraries} \vspace*{\captionSpace{}} \label{table:serial-libs}
\input manifest/serial-libs
\vspace*{\tabSpaceBot{}}
\end{table}

% Parallel libs
\begin{table}[h!]
\caption{\bf Parallel Libraries} \vspace*{\captionSpace{}} \label{table:parallel-libs}
\input manifest/parallel-libs
\vspace*{\tabSpaceBot{}}
\end{table}

% Parallel libs (2)
\begin{table}[h!]
\caption*{Table~\ref{table:parallel-libs} (cont): {\bf Parallel Libraries} \vspace*{\captionSpace{}} }
\input manifest/parallel-libs2
\vspace*{\tabSpaceBot{}}
\end{table}

\input{common/signature}

\section{TESTING PR 1878}
THIS IS A TEMPORARY CHANGE TO TEST THE BUILD OF DOCS WHEN A .TEX FILE IS MODIFIED

\end{document}

